\chapter{Fundamentação teórica}
%%%%%%%%%%%%%%%%%%%%%%%%%%%%%%%%%%%%%%%%%%%%%%%%%%%%%%%%%%%%%%%%%%%%%%
Neste capitulo, apresentamos uma síntese da revisão bibliográfica realizada para embasar a metodologia adotada no desenvolvimento e simulação do ADPLL. São abordados os conceitos básicos de um sintetizador de frequência, o protocolo de comunicação \textit{Bluetooth} e a importância dos circuitos e dispositivos CMOS. Esses tópicos são fundamentais para compreender o funcionamento do ADPLL e sua aplicação em sistemas de comunicação sem fio. A revisão bibliográfica oferece uma base sólida para o desenvolvimento do projeto, fornecendo os conhecimentos necessários para explorar as características, desempenho e aplicações do ADPLL.

%%%%%%%%%%%%%%%%%%%%%%%%%%%%%%%%%%%%%%%%%%%%%%%%%%%%%%%%%%%%%%%%%%%%%%
\section{Sintetizador de frequência}
%%%%%%%%%%%%%%%%%%%%%%%%%%%%%%%%%%%%%%%%%%%%%%%%%%%%%%%%%%%%%%%%%%%%%%
Em sistemas de comunicação sem fio a presença de um circuito sintetizador de frequência é essencial. O circuito sintetizador de frequência é responsável por gerar a frequência central de um canal em um sistema de comunicação de Rádio Frequência (RF). Cada canal possui uma faixa de frequência especifica de operação, desta forma o circuito do Sintetizador deve ser capaz de permitir ajustes de frequência pequenos.

O circuito Sintetizador de frequência gera as frequências necessária como um múltiplo de uma referência do Oscilador de Cristal Controlado por Temperatura (TXCO, do inglês \textit{Temperature-Controlled Crystal Oscillator}). O TXCO papel fundamental na performance do sintetizador, é responsavél por fornece uma frequência estável e precisa e com baixo valor de ruido de fase. De acordo com \cite{lascari2000accurate} negligenciar seus efeitos em um sintetizador pode acarretar em resultados inesperados após a concepção do circuito.

O processo de síntese de frequência ocorre através de técnicas de geração e mistura de sinais. Primeiramente, a referência de TXCO fornece uma frequência estável e precisa. Em seguida, o sintetizador de frequência utiliza circuitos internos, como divisores de frequência e circuitos de fase, para manipular e multiplicar a frequência da referência, produzindo assim a frequência desejada.


%%%%%%%%%%%%%%%%%%%%%%%%%%%%%%%%%%%%%%%%%%%%%%%%%%%%%%%%%%%%%%%%%%%%%%
\section{PLL}
%%%%%%%%%%%%%%%%%%%%%%%%%%%%%%%%%%%%%%%%%%%%%%%%%%%%%%%%%%%%%%%%%%%%%%
PLL (\textit{Phase-Locked-Lopp}) é um circuito Sintetizador de frequência comumente utilizado. PLL é composto por diversos blocos, alguns deles são: osciladores controlados por tensão VCO (\textit{Voltage-Controlled Oscillator}), divisores programáveis, comparadores de fase DFF (Detectores de Fase e Frequência), bombas de carga CP (\textit{Charge Pump}) e Filtros Passa Baixas LPF (\textit{Low Pass Filters}). 

Os blocos que compõem um PLL são apresentados na Figura \ref{fig:pll_blocks}.
A utilização de uma realimentação negativa no circuito permite um controle tanto de frequência como de fase para a saída.

\begin{figure}[h!]
	\caption{Diagrama de blocos de um PLL.}
	\begin{center}
		\includegraphics[scale=0.6]{img/pll_blocos.png}
	\end{center}
	\fonte{\citeonline{barrett_1999_fractionalintegern}}
	\label{fig:pll_blocks}
\end{figure}

O PLL utiliza o VCO como elemento central. O sinal de saída do VCO, dividido por um fator N, é comparado com a frequência de referência do TXCO, dividida por R, pelo  \textit{Phase-Detector}. Após a comparação, o sinal resultante passa pelo\textit{Loop filter}, responsável por eliminar ruídos e interferências. A saída do \textit{Loop filter} é uma tensão que controla a tensão aplicada ao VCO, permitindo ajustar e manter a frequência de saída em sincronia com a frequência de referência \cite{barrett_1999_fractionalintegern}.

Em condições normais, um PLL fornece uma frequência com extrema precisão, no entanto, o tempo de aquisição pode ser longo devido ao processo do detector de fase e frequência em avaliar e gerar sinais com base nas diferenças em relação à referência. Esse tempo de aquisição é especialmente crucial em aplicações de comunicações sem fio que envolvem técnicas como salto de canal (\textit{frequency hopping}), como é o caso do protocolo \textit{Bluetooth}. Nessas situações, a capacidade do PLL de se sincronizar rapidamente com frequências variáveis é essencial para garantir uma transição suave entre os canais e evitar perdas de dados ou conexão.
%%%%%%%%%%%%%%%%%%%%%%%%%%%%%%%%%%%%%%%%%%%%%%%%%%%%%%%%%%%%%%%%%%%%%%
\subsection{Integer-N PLL}
%%%%%%%%%%%%%%%%%%%%%%%%%%%%%%%%%%%%%%%%%%%%%%%%%%%%%%%%%%%%%%%%%%%%%%

PLLs convencionais também conhecidos como \textit{Integer-N PLL} são capazes de gerar apenas frequências de valores N vezes a frequência do TXCO, onde N é um valor inteiro, desta forma a resolução de frequência é definida pela frequência de referência utilizada. 

A frequência de saída é definida como:

\begin{equation}
	F_{VCO} = N \cdot F_{ref}
	\label{eq:fvco_integer_PLL}
\end{equation}



%%%%%%%%%%%%%%%%%%%%%%%%%%%%%%%%%%%%%%%%%%%%%%%%%%%%%%%%%%%%%%%%%%%%%%
\subsection{Fractional-N PLL}
%%%%%%%%%%%%%%%%%%%%%%%%%%%%%%%%%%%%%%%%%%%%%%%%%%%%%%%%%%%%%%%%%%%%%%
Em um \textit{fractional-N} PLL a frequência de saída pode ser ajustada como uma fração da frequência de referência. Esse ajuste fracional é necessário em sistemas de comunicação para o ajuste correto da frequência central de canal utilizado. 

\textit{Fractional-N} PLL utiliza uma topologia similar ao do \textit{Integer-N} PLL, com adição de um acumulador, uma maquina de alterna o divisor entre (N e N+1) durante um estado bloqueado. Esta variação faz com que a média torne-se um valor fracional entre N e N+1, proporcionando um ajuste de frequência também fracional. 

A frequência de saída é definida como:

\begin{equation}
	F_{VCO} = (N + F) \cdot F_{ref}
	\label{eq:fvco_fractional_PLL}
\end{equation}
Onde, $F$ é $0$ ou $1$. 

%%%%%%%%%%%%%%%%%%%%%%%%%%%%%%%%%%%%%%%%%%%%%%%%%%%%%%%%%%%%%%%%%%%%%%
\section{ADPLL}
%%%%%%%%%%%%%%%%%%%%%%%%%%%%%%%%%%%%%%%%%%%%%%%%%%%%%%%%%%%%%%%%%%%%%%
O ADPLL \textit{All Digital Phase-Locked-Loop} é um circuito Sintetizador de frequência que ao contrário dos PLLs convencionais é um puramente digital. Em um PLL tradicional o \textit{Loop-Filter} ocupa mais de 50\% da área do chip, enquanto que no ADPLL por ser um circuito digital não necessita de grandes componentes, capacitores, resistores e indutores, reduzindo em grande parte a área ocupada.

A utilização de circuito digital se da devido a miniaturização da tecnologia CMOS, permitindo maiores velocidades, maiores frequências, e assim, propiciando o uso de \textit{design} de circuitos digitais em RF. Seu uso traz inúmeros benefícios, entre elas a possibilidade de parametrização do \textit{Loop-Filter} para ajuste de frequência conforme desejado, e sendo o ADPLL totalmente digital não necessita circuitos auxiliar de conversão do sinal digital para analógico ou vice-versa, sendo uma grande vantagem.

Tanto uma onda sinusoidal quanto uma onda retangular podem ser utilizadas como a frequência de referência do circuito. No entanto, devido ao fato de o circuito ser digital ser controlado pelas transições do \textit{clock}, é comum utilizar uma onda retangular para análise. 
%Neste trabalho, optou-se por realizar uma análise focada nas transições do \textit{clock} para estudo e desenvolvimento. 

A Figura \ref{fig:adpll_block_diagram} mostra um diagrama de blocos de um ADPLL baseado em transições de \textit{clock}, baseado no estudo de \cite{andersson2010modeling} com algumas modificações. O ADPLL é composto de 4 blocos principais,  \textit{Digital Controlled Oscillator} (DCO), \textit{Time to Digital Converter} (TDC), \textit{Phase Detector} (PD) e o \textit{ digital Loop Filter} (LF). 
%Nas subseções seguintes serão apresentados com mais detalhes cada um deles. 

\begin{figure}[h!]
	\caption{Diagrama de blocos de um ADPLL.}
	\begin{center}
		\includegraphics[scale=0.6]{img/blocos_ADPLL.pdf}
	\end{center}
%	\fonte{\citeonline{staszewski2006all}}
	\label{fig:adpll_block_diagram}
\end{figure}

O circuito do DCO é responsável por gerar o sinal de saída do sintetizador. Ele consiste em um indutor fixo e um conjunto de capacitores programáveis que formam um circuito ressonante LC. A frequência de saída é definida pelo FCW (\textit{Frequency Command Word}), que é um valor múltiplo da frequência de referência, conforme expresso na Equação \ref{eq:fvco_adpll}, podendo FCW ser um valor fracional.

\begin{equation}
	F_{VCO} = FCW \cdot F_{ref}
	\label{eq:fvco_adpll}
\end{equation}

Por outro lado, o TDC realiza a medição da diferença de tempo entre as bordas de \textit{clock} do sinal do DCO e uma borda de referência. Esse valor é acumulado da mesma forma que o FCW, a cada borda de \textit{clock}. O detector de fase compara as diferenças entre os acumuladores, que é então utilizado pelo\textit{Loop Filter} para controlar os capacitores do DCO. Essa ação de controle resulta no ajuste da frequência do sinal de saída, permitindo aumentá-la ou reduzi-la conforme necessário.




%%%%%%%%%%%%%%%%%%%%%%%%%%%%%%%%%%%%%%%%%%%%%%%%%%%%%%%%%%%%%%%%%%%%%%
%\textit{Energy Harvesting} 
%%%%%%%%%%%%%%%%%%%%%%%%%%%%%%%%%%%%%%%%%%%%%%%%%%%%%%%%%%%%%%%%%%%%%%

%%%%%%%%%%%%%%%%%%%%%%%%%%%%%%%%%%%%%%%%%%%%%%%%%%%%%%%%%%%%%%%%%%%%%%
\subsection{DCO}
%%%%%%%%%%%%%%%%%%%%%%%%%%%%%%%%%%%%%%%%%%%%%%%%%%%%%%%%%%%%%%%%%%%%%%

%%%%%%%%%%%%%%%%%%%%%%%%%%%%%%%%%%%%%%%%%%%%%%%%%%%%%%%%%%%%%%%%%%%%%%
\subsection{Loop Filter}
%%%%%%%%%%%%%%%%%%%%%%%%%%%%%%%%%%%%%%%%%%%%%%%%%%%%%%%%%%%%%%%%%%%%%%

%%%%%%%%%%%%%%%%%%%%%%%%%%%%%%%%%%%%%%%%%%%%%%%%%%%%%%%%%%%%%%%%%%%%%%
\subsection{TDC}
%%%%%%%%%%%%%%%%%%%%%%%%%%%%%%%%%%%%%%%%%%%%%%%%%%%%%%%%%%%%%%%%%%%%%%

%%%%%%%%%%%%%%%%%%%%%%%%%%%%%%%%%%%%%%%%%%%%%%%%%%%%%%%%%%%%%%%%%%%%%%
\subsection{Ruído de Fase}
%%%%%%%%%%%%%%%%%%%%%%%%%%%%%%%%%%%%%%%%%%%%%%%%%%%%%%%%%%%%%%%%%%%%%%


%%%%%%%%%%%%%%%%%%%%%%%%%%%%%%%%%%%%%%%%%%%%%%%%%%%%%%%%%%%%%%%%%%%%%%
\section{Trabalhos correlatos}
%%%%%%%%%%%%%%%%%%%%%%%%%%%%%%%%%%%%%%%%%%%%%%%%%%%%%%%%%%%%%%%%%%%%%%


%%%%%%%%%%%%%%%%%%%%%%%%%%%%%%%%%%%%%%%%%%%%%%%%%%%%%%%%%%%%%%%%%%%%%%

