\chapter{Fundamentação teórica}
%%%%%%%%%%%%%%%%%%%%%%%%%%%%%%%%%%%%%%%%%%%%%%%%%%%%%%%%%%%%%%%%%%%%%%
Neste capitulo, apresentamos uma síntese da revisão bibliográfica realizada para embasar a metodologia adotada no desenvolvimento e simulação do ADPLL. São abordados os conceitos básicos de um sintetizador de frequência, o protocolo de comunicação \textit{Bluetooth} e a importância dos circuitos e dispositivos CMOS, e alguns conceitos de comunicações sem fio. Esses tópicos são fundamentais para compreender o funcionamento do ADPLL e sua aplicação em sistemas de comunicação sem fio. A revisão bibliográfica oferece uma base sólida para o desenvolvimento do projeto, fornecendo os conhecimentos necessários para explorar as características, desempenho e aplicações do ADPLL.


%%%%%%%%%%%%%%%%%%%%%%%%%%%%%%%%%%%%%%%%%%%%%%%%%%%%%%%%%%%%%%%%%%%%%%
\section{Sistemas de Comunicação sem fio}
%%%%%%%%%%%%%%%%%%%%%%%%%%%%%%%%%%%%%%%%%%%%%%%%%%%%%%%%%%%%%%%%%%%%%%
%Sistemas de comunicação sem fio, são capazes de transportar informações sem a necessidade de conexões físicas. Comunicações sem fio (do inglês, \textit{Wirelless}), são ondas eletromagnéticas em um range entre $3kHz$ à $300GHz$, que se propagam na velocidade da luz sem a necessidade de ar para se locomover.
%
%Desde as primeiras transmissões de voz por radio na década de 1920 até os dias atuais houve muitas evoluções, estando hoje comunicações em fio presentes em tudo no dia a dia, desde televisão, rádio, celulares, internet (WIFI) e muito mais \cite{dowla2003handbook}. 
%
%Uma informação é transmitida por rádio frequência (RF), por meio de uma técnica chamada modulação. \cite{engtadeu2011} define que "Modulação é um processo que consiste em se alterar uma característica da onda portadora, proporcionalmente ao sinal modulante".  De forma básica consiste em empacotar a informação( sinal modulante), (áudio ou dados digitais), que é de baixa frequência , em uma onda de alta frequência (onda Portadora), tendo como resultado um sinal modulado,  e então, através de uma antena transmitir pelo ar. Existem inúmeros modelos de modulação que podem ser utilizados, cada um com características diferentes que visam aumentar alcance, imunidade a ruido, evitar colisão entre outros.
%
%Um dispositivo de comunicação sem fio é composto de dois circuitos, um receptor, para receber e demodular a informação e outro transmissor para enviar a informação modulada. Um dispositivo pode ser composto apenas de um transmissor ou receptor conforme a aplicação, e nos casos que possui ambos também é chamado de transceptor. A Figura \ref{fig:tranceiver_structure} apresenta a estrutura básica de um transceptor onde cada bloco tem uma função especifica.


Os sistemas de comunicação sem fio proporcionam a transferência de informações sem a necessidade de conexões físicas. Essas comunicações, conhecidas como \textit{wireless}, utilizam ondas eletromagnéticas em um amplo espectro de frequências, variando de $3kHz$ a $300GHz$, propagando-se na velocidade da luz pelo espaço sem depender de um meio físico específico.

Desde as primeiras transmissões de voz por rádio na década de 1920 até os dias atuais, houve significativas evoluções nesse campo, com as comunicações sem fio se tornando uma parte integrante do cotidiano, presente em tecnologias como televisão, rádio, celulares, internet (Wi-Fi) e muito mais \cite{dowla2003handbook}.

A transmissão de informações por rádio frequência (RF) é realizada por meio de uma técnica chamada modulação. De maneira geral, a modulação envolve alterar uma característica da onda portadora de acordo com o sinal modulante, que carrega a informação a ser transmitida, seja áudio ou dados digitais \cite{engtadeu2011}. Essa técnica empacota a informação de baixa frequência em uma onda de alta frequência, resultando em um sinal modulado que pode ser transmitido pelo ar através de uma antena. Existem inúmeros modelos de modulação, cada um com características específicas que visam aumentar o alcance, a imunidade a ruídos, evitar colisões entre sinais, entre outros aspectos.

Um dispositivo de comunicação sem fio é composto por dois circuitos principais: um receptor, responsável por receber e demodular a informação, e um transmissor, encarregado de enviar a informação modulada. Dependendo da aplicação, um dispositivo pode ser composto apenas por um transmissor ou receptor, enquanto em outros casos, que envolvem ambas as funções, ele é denominado transceptor. A Figura \ref{fig:tranceiver_structure} apresenta a estrutura básica de um transceptor, onde cada bloco desempenha uma função específica.
\begin{figure}[h!]
	\caption{Estrutura de um transceptor.}
	\begin{center}
		\includegraphics[scale=0.6]{img/tranceiver_structure.png}
	\end{center}
	\fonte{\citeonline{srakar_tranceiver_2023}.}
	\label{fig:tranceiver_structure}
\end{figure}


%%%%%%%%%%%%%%%%%%%%%%%%%%%%%%%%%%%%%%%%%%%%%%%%%%%%%%%%%%%%%%%%%%%%%%
\section{CMOS}
%%%%%%%%%%%%%%%%%%%%%%%%%%%%%%%%%%%%%%%%%%%%%%%%%%%%%%%%%%%%%%%%%%%%%%
CMOS (\textit{Complementary Metal-Oxide-Semiconductor}) é um processo de fabricação que utiliza silício para criar transistores de efeito de campo (MOSFETs) do tipo p e do tipo n, chamados de PMOS e NMOS, respectivamente. Amplamente utilizado na produção de chips de circuitos integrados, como microprocessadores, microcontroladores e chips de memória, o CMOS também é empregado em circuitos analógicos, como sensores de imagem e dispositivos de comunicação com e sem fio. Sua característica distintiva é o baixo consumo de energia, tornando-o especialmente adequado para dispositivos IoT.

Além disso, o CMOS oferece vantagens em relação a outras tecnologias, como operação em baixas tensões e uma evolução avançada que permite maior densidade de transistores por unidade de área . O tamanho dos transistores é determinado pelos valores de comprimento (L) e largura (W) Figura \ref{fig:nmos_structure}, permitindo um controle mais preciso da corrente que circula por eles\cite{designcmosrazavi2016}. Atualmente, circuitos integrados podem ser fabricados em escala de $3nm$ de acordo com \cite{tsmc}.

\begin{figure}[h!]
	\caption{Estrutura de um NMOS}
	\begin{center}
		\includegraphics[scale=0.6]{img/nmos_structure.png}
	\end{center}
	\fonte{\citeonline{designcmosrazavi2016}.}
	\label{fig:nmos_structure}
\end{figure}

%%%%%%%%%%%%%%%%%%%%%%%%%%%%%%%%%%%%%%%%%%%%%%%%%%%%%%%%%%%%%%%%%%%%%%
\section{Bluetooth}
%%%%%%%%%%%%%%%%%%%%%%%%%%%%%%%%%%%%%%%%%%%%%%%%%%%%%%%%%%%%%%%%%%%%%%
O \textit{Bluetooth} é um padrão de tecnologia sem fio de curto alcance, amplamente utilizado para a troca de dados entre dispositivos fixos e móveis em distâncias curtas. 

Atualmente, milhões de pessoas em todo o mundo utilizam dispositivos \textit{Bluetooth} para transferir dados, como música, fotos e vídeos, no dia a dia. Além disso, o \textit{Bluetooth} tem sido cada vez mais empregado em dispositivos IoT, especialmente com a implementação do BLE (\textit{Bluetooth Low Energy}), que oferece maior alcance e menor consumo de energia em relação ao \textit{Bluetooth} padrão. Estima-se que em 2027 existirão cerca de 7,6 bilhões de dispositivos \textit{Bluetooth} ativos em todo o mundo \cite{BluetoothSite}.


Gerenciado pelo \textit{Bluetooth Special Interest Group} (SIG), o padrão opera no espectro de frequência $2,4GHz$. Na Tabela \ref{tab:bluetooth_specification} é apresentado algumas definições de comunicação \textit{Bluetooth} clássico e do BLE.

\begin{table}[h!]
	\caption{Definições de \textit{Bluetooth} clássico e BLE}
	
	\label{tab:bluetooth_specification}
	\resizebox{\columnwidth}{!}{%
		\begin{tabular}{lll}
			\hline
			& \textbf{\textit{Bluetooth} Low Energy (BLE)}      & \textbf{\textit{Bluetooth} Clássico}               \\ \hline
			\textbf{Banda de frequência} & 2.4 GHz Banda ISM  (2.402 – 2.480 GHz)    & 2.4 GHz Banda ISM  (2.402 – 2.480 GHz)    \\ \hline
			\textbf{Canais}              & 40 canais com 2 MHz de espaçamento        & 79 canais com 1 MHz de espaçamento       \\ \hline
			\textbf{Modulação}           & GFSK                                     & GFSK, $\pi$/4 DQPSK, 8DPSK                   \\ \hline
			\textbf{Uso do Canal}        & \textit{Frequency-Hopping Spread Spectrum} (FHSS) & \textit{Frequency-Hopping Spread Spectrum} (FHSS) \\ \hline
			\textbf{Data Rate (DR)}      & Até 2 Mb/s                                & Até 3 Mb/s                                \\ \hline
			\textbf{Sensibilidade (RX)}  & -82dBm com DR=120 kb/s                    & -70 dBm                                  \\ \hline
			\textbf{Topologia de comunicação} & \begin{tabular}[c]{@{}l@{}}Ponto a ponto\\ \textit{Broadcast}\\ \textit{Mesh}\end{tabular} & Ponto a ponto  \\ \hline
		\end{tabular}%		
	}
	\fonte{Adaptado de \citeonline{BluetoothSite}.}
\end{table}


%%%%%%%%%%%%%%%%%%%%%%%%%%%%%%%%%%%%%%%%%%%%%%%%%%%%%%%%%%%%%%%%%%%%%%
\section{Sintetizador de frequência}
%%%%%%%%%%%%%%%%%%%%%%%%%%%%%%%%%%%%%%%%%%%%%%%%%%%%%%%%%%%%%%%%%%%%%%
Em sistemas de comunicação sem fio a presença de um circuito sintetizador de frequência é essencial. O circuito sintetizador de frequência é responsável por gerar a frequência central de um canal em um sistema de comunicação de Rádio Frequência (RF). Cada canal possui uma faixa de frequência especifica de operação, desta forma o circuito do Sintetizador deve ser capaz de permitir ajustes de frequência pequenos.

O circuito Sintetizador de frequência gera as frequências necessária como um múltiplo de uma referência do Oscilador de Cristal Controlado por Temperatura (TXCO, do inglês \textit{Temperature-Controlled Crystal Oscillator}). O TXCO papel fundamental na performance do sintetizador, é responsavél por fornece uma frequência estável e precisa e com baixo valor de ruido de fase. De acordo com \cite{lascari2000accurate} negligenciar seus efeitos em um sintetizador pode acarretar em resultados inesperados após a concepção do circuito.

O processo de síntese de frequência ocorre através de técnicas de geração e mistura de sinais. Primeiramente, a referência de TXCO fornece uma frequência estável e precisa. Em seguida, o sintetizador de frequência utiliza circuitos internos, como divisores de frequência e circuitos de fase, para manipular e multiplicar a frequência da referência, produzindo assim a frequência desejada.


%%%%%%%%%%%%%%%%%%%%%%%%%%%%%%%%%%%%%%%%%%%%%%%%%%%%%%%%%%%%%%%%%%%%%%
\section{PLL}
%%%%%%%%%%%%%%%%%%%%%%%%%%%%%%%%%%%%%%%%%%%%%%%%%%%%%%%%%%%%%%%%%%%%%%
PLL (\textit{Phase-Locked-Lopp}) é um circuito Sintetizador de frequência comumente utilizado. PLL é composto por diversos blocos, alguns deles são: osciladores controlados por tensão VCO (\textit{Voltage-Controlled Oscillator}), divisores programáveis, comparadores de fase DFF (Detectores de Fase e Frequência), bombas de carga CP (\textit{Charge Pump}) e Filtros Passa Baixas LPF (\textit{Low Pass Filters}). 

Os blocos que compõem um PLL são apresentados na Figura \ref{fig:pll_blocks}.
A utilização de uma realimentação negativa no circuito permite um controle tanto de frequência como de fase para a saída.

\begin{figure}[h!]
	\caption{Diagrama de blocos de um PLL.}
	\begin{center}
		\includegraphics[scale=0.6]{img/pll_blocos.png}
	\end{center}
	\fonte{\citeonline{barrett_1999_fractionalintegern}.}
	\label{fig:pll_blocks}
\end{figure}

O PLL utiliza o VCO como elemento central. O sinal de saída do VCO, dividido por um fator N, é comparado com a frequência de referência do TXCO, dividida por R, pelo  \textit{Phase-Detector}. Após a comparação, o sinal resultante passa pelo\textit{Loop filter}, responsável por eliminar ruídos e interferências. A saída do \textit{Loop filter} é uma tensão que controla a tensão aplicada ao VCO, permitindo ajustar e manter a frequência de saída em sincronia com a frequência de referência \cite{barrett_1999_fractionalintegern}.

Em condições normais, um PLL fornece uma frequência com extrema precisão, no entanto, o tempo de aquisição pode ser longo devido ao processo do detector de fase e frequência em avaliar e gerar sinais com base nas diferenças em relação à referência. Esse tempo de aquisição é especialmente crucial em aplicações de comunicações sem fio que envolvem técnicas como salto de canal (\textit{frequency hopping}), como é o caso do protocolo \textit{Bluetooth}. Nessas situações, a capacidade do PLL de se sincronizar rapidamente com frequências variáveis é essencial para garantir uma transição suave entre os canais e evitar perdas de dados ou conexão.
%%%%%%%%%%%%%%%%%%%%%%%%%%%%%%%%%%%%%%%%%%%%%%%%%%%%%%%%%%%%%%%%%%%%%%
\subsection{Integer-N PLL}
%%%%%%%%%%%%%%%%%%%%%%%%%%%%%%%%%%%%%%%%%%%%%%%%%%%%%%%%%%%%%%%%%%%%%%

PLLs convencionais também conhecidos como \textit{Integer-N PLL} são capazes de gerar apenas frequências de valores N vezes a frequência do TXCO, onde N é um valor inteiro, desta forma a resolução de frequência é definida pela frequência de referência utilizada. 

A frequência de saída é definida como:

\begin{equation}
	F_{VCO} = N \cdot F_{ref}
	\label{eq:fvco_integer_PLL}
\end{equation}



%%%%%%%%%%%%%%%%%%%%%%%%%%%%%%%%%%%%%%%%%%%%%%%%%%%%%%%%%%%%%%%%%%%%%%
\subsection{Fractional-N PLL}
%%%%%%%%%%%%%%%%%%%%%%%%%%%%%%%%%%%%%%%%%%%%%%%%%%%%%%%%%%%%%%%%%%%%%%
Em um \textit{fractional-N} PLL a frequência de saída pode ser ajustada como uma fração da frequência de referência. Esse ajuste fracional é necessário em sistemas de comunicação para o ajuste correto da frequência central de canal utilizado. 

\textit{Fractional-N} PLL utiliza uma topologia similar ao do \textit{Integer-N} PLL, com adição de um acumulador, uma maquina de alterna o divisor entre (N e N+1) durante um estado bloqueado. Esta variação faz com que a média torne-se um valor fracional entre N e N+1, proporcionando um ajuste de frequência também fracional. 

A frequência de saída é definida como:

\begin{equation}
	F_{VCO} = (N + F) \cdot F_{ref}
	\label{eq:fvco_fractional_PLL}
\end{equation}
Onde, $F$ é $0$ ou $1$. 

%%%%%%%%%%%%%%%%%%%%%%%%%%%%%%%%%%%%%%%%%%%%%%%%%%%%%%%%%%%%%%%%%%%%%%
\section{ADPLL}
%%%%%%%%%%%%%%%%%%%%%%%%%%%%%%%%%%%%%%%%%%%%%%%%%%%%%%%%%%%%%%%%%%%%%%
%O ADPLL \textit{All Digital Phase-Locked-Loop} é um circuito Sintetizador de frequência que ao contrário dos PLLs convencionais é um puramente digital. Em um PLL tradicional o \textit{Loop-Filter} ocupa mais de 50\% da área do chip, enquanto que no ADPLL por ser um circuito digital não necessita de grandes componentes, capacitores, resistores e indutores, reduzindo em grande parte a área ocupada.
%
%A utilização de circuito digital se da devido a miniaturização da tecnologia CMOS, permitindo maiores velocidades, maiores frequências, e assim, propiciando o uso de \textit{design} de circuitos digitais em RF. Seu uso traz inúmeros benefícios, entre elas a possibilidade de parametrização do \textit{Loop-Filter} para ajuste de frequência conforme desejado, e sendo o ADPLL totalmente digital não necessita circuitos auxiliar de conversão do sinal digital para analógico ou vice-versa, sendo uma grande vantagem.


O ADPLL (\textit{All Digital Phase-Locked-Loop}) é um sintetizador de frequência que se diferencia dos PLLs convencionais por ser um circuito puramente digital. Enquanto um PLL tradicional requer componentes analógicos, como capacitores, resistores e indutores, o ADPLL aproveita os benefícios da miniaturização da tecnologia CMOS, permitindo maiores velocidades e frequências, além de reduzir significativamente a área ocupada no chip.

Para \cite{staszewski2006all} a tecnologia nanométrica do CMOS traz um novo paradigma, a resolução do domínio do tempo de uma transição de borda de sinal digital é superior à resolução de tensão de sinais analógicos. Desta forma o ADPLL pode ser analisado apenas pelas transições dos sinais digitais.

A natureza digital do ADPLL oferece vantagens adicionais, como a parametrização do \textit{Loop-Filter} para ajuste de frequência conforme necessário. Além disso, não são necessários circuitos auxiliares para conversão entre sinais digitais e analógicos, o que representa uma economia de recursos e simplificação do projeto.

%Tanto uma onda sinusoidal quanto uma onda retangular podem ser utilizadas como a frequência de referência do circuito. No entanto, é comum usar uma onda retangular para análise, visto que o circuito é sincronizado e controlado pelas transições do \textit{clock} de referência (FREF), proporcionando uma análise rápida e precisa do comportamento do ADPLL.

A Figura \ref{fig:adpll_block_diagram} mostra o diagrama de blocos de um ADPLL no domínio do tempo, ou seja considerando as transições dos sinais de (FREF) e saída do circuito (DCO). O ADPLL é composto de 4 blocos principais,  \textit{Digital Controlled Oscillator} (DCO), \textit{Time to Digital Converter} (TDC), \textit{Phase Detector} (PD) e o \textit{ digital Loop Filter} (LF). Nas subseções seguintes serão apresentados com mais detalhes cada um dos sub-blocos. 

\begin{figure}[h!]
	\caption{Diagrama de blocos de um ADPLL.}
	\begin{center}
		\includegraphics[scale=1]{img/blocos_ADPLL.png}
	\end{center}
%	\fonte{\citeonline{staszewski2006all}}
\fonte{Adaptado de \citeonline{andersson2010modeling}.}
	\label{fig:adpll_block_diagram}
\end{figure}

O circuito do DCO é responsável por gerar o sinal de saída do sintetizador. Ele consiste em um indutor fixo e um conjunto de capacitores programáveis que formam um circuito ressonante LC. A frequência de saída é definida pelo FCW (\textit{Frequency Command Word}), pode ser um valor fracionado ou inteiro da frequência de referência, conforme expresso na Equação \ref{eq:fvco_adpll}.

\begin{equation}
	F_{VCO} = FCW \cdot F_{ref}
	\label{eq:fvco_adpll}
\end{equation}

Por outro lado, o TDC realiza a medição da diferença de tempo entre as bordas de \textit{clock} do sinal do DCO e uma borda de referência, , acumulando esse valor a cada transição do\textit{clock}, da mesma forma que FCW. O detector de fase compara as diferenças entre os acumuladores, que é então utilizado pelo\textit{Loop Filter} para controlar os capacitores do DCO. Essa ação de controle resulta no ajuste da frequência do sinal de saída, permitindo aumentá-la ou reduzi-la conforme necessário.



%É comumente utilizado uma onda retangular para análise, na Figura \ref{fig:adpll_block_diagram}  o circuito é sincronizado e controlado pelas transições do \textit{clock} de referência (FREF), o que permite uma análise rápida e precisa do comportamento.

%Neste trabalho, optou-se por realizar uma análise focada nas transições do \textit{clock} para estudo e desenvolvimento. 


%%%%%%%%%%%%%%%%%%%%%%%%%%%%%%%%%%%%%%%%%%%%%%%%%%%%%%%%%%%%%%%%%%%%%%
%\textit{Energy Harvesting} 
%%%%%%%%%%%%%%%%%%%%%%%%%%%%%%%%%%%%%%%%%%%%%%%%%%%%%%%%%%%%%%%%%%%%%%

%%%%%%%%%%%%%%%%%%%%%%%%%%%%%%%%%%%%%%%%%%%%%%%%%%%%%%%%%%%%%%%%%%%%%%
\subsection{DCO}
%%%%%%%%%%%%%%%%%%%%%%%%%%%%%%%%%%%%%%%%%%%%%%%%%%%%%%%%%%%%%%%%%%%%%%
O DCO é o elemento principal do ADPLL, converte o \textit{Oscilator Tunning Word}, ($OTW$), em um sinal periódico onde a frequência $f$ é definida em função de sua entrada.
\begin{equation}
	f = f(OTW)
	\label{eq:f_OTW}
\end{equation}

O DCO é formado por um circuito tanque LC, um indutor fixo e capacitores programáveis, em ressonância a frequência é definida de acordo com a equação \ref{eq:f_LC} onde $C_{tot}$ é a soma de todas capacitâncias.

\begin{equation}
	f_{out} = \frac{1}{2 \pi \sqrt{L \cdot C_{tot}}}
	\label{eq:f_LC}
\end{equation}

Transistores em modo varactor são utilizado como capacitores programáveis, pois possuem um ajuste muito pequeno de frequência em relação a varactores de diodo. Quanto menor o valor da capacitância menor é o  passo de frequência de ajuste do DCO, permitindo um ajuste mais fino e melhor performance. 

Transistores do tipo PMOS são comumente utilizados como varactores devido a suas propriedades de isolação do poço. Com os terminais dreno, fonte e corpo (D=S=B) interligados permite que a capacitância seja controlada pelo nível de tensão $VG$ conforme região de operação do transistor, limitada pela capacitância do óxido $(COX)$. A Figura \ref{fig:curva_COX_vbg} mostra a curva de capacitância

\begin{figure}[h!]
	\caption{Curva de resposta do varactor D=S=B em relação a $VG$ }
	\begin{center}
		\includegraphics[scale=0.6]{img/curva_COX_vbg.png}
	\end{center}
	\fonte{\citeonline{lucasvco2022}.}
	\label{fig:curva_COX_vbg}
\end{figure}

O DCO deve ser capaz de gerar um range de frequências com ajuste fino para atender às necessidades de modulação do transceptor. Por exemplo, para o \textit{Bluetooth}, que possui um range de frequências entre $2.402GHz$ e $2.480GHz$, usar apenas 8 bits, um conjunto de 256 capacitores programáveis, resultaria em um passo de frequência muito grosso de aproximadamente $304,67kHz$, o que não é viável para aplicações sem fio e tipos de modulações utilizadas.

Para superar essa limitação, a solução adotada é dividir o banco de capacitores em três partes: modo de Processo-Tensão-Temperatura (PVT), modo de aquisição (ACQ) e modo de caminhada (TRK). Esses modos permitem ajustar o DCO de maneira mais precisa. De acordo com \cite{staszewski2006all}, uma boa relação entre tamanho físico e menor passo de frequência é alcançada atribuindo 8 bits para o modo PVT, 8 bits para o modo ACQ e 6 bits para o modo TRK.

%
%O DCO deve ser capaz de gerar um range de frequências e um espaçamento entre elas que atenda as especificações de modulação que será utilizado no transceptor. Para o \textit{Bluetooth } que possui um range de frequências entre $2.402GHz$ e $2.480GHz$, se utilizar 8 bits, conjunto de 256 capacitores programáveis, o passo de frequência seria $(2,480GHz - 2,402GHz)/2^8 = 304,67kHz$, o que é muito grosso para aplicações sem fio. Por outro lado se considerar um passo de frequência de $1kHz$ para a banda toda do \textit{Bluetooth }, seriam necessário 17 bits, sendo inviável de fabricação pela extrema dificuldade de casamento dos inúmeros capacitores no \textit{layout}.
%
%
%A alternativa é a divisão do banco de capacitores em três partes, sendo apenas uma acionada por vez. Os modos são chamados de modo de Processo-Tensão-Temperatura  (PVT), modo de aquisição (ACQ) e modo de caminhada (TRK). Para uma boa relação entre tamanho físico e o menor passo de frequência\cite{staszewski2006all} sugere que cada banco possua 8 bits em modo PVT, 8 bits no modo ACQ e 6 bits em TRK.  O modo PVT é o primeiro ser setado, compensando as variações de processo, temperatura e tensão, centralizando o oscilador mais perto da frequência desejada, mas não exata, com variações entre $1$ a $2MHz$. O segundo modo é o ACQ, ajustando a frequência para dentro da banda de interesse, com uma resolução de $\pm 500kHz$ utilizando 8 bits neste banco de capacitores. Por fim o modo TRK é a menor resolução para ajustar a frequência exata utilizada na modulação do sinal dentro do canal utilizado.

%A solução adotada é dividir o banco de capacitores em três partes, ativando apenas uma delas por vez. Esses modos são conhecidos como modo de Processo-Tensão-Temperatura (PVT), modo de aquisição (ACQ) e modo de caminhada (TRK). Para alcançar um equilíbrio adequado entre tamanho físico e menor passo de frequência, de acordo com \cite{staszewski2006all}, sugere-se que cada banco possua 8 bits no modo PVT, 8 bits no modo ACQ e 6 bits no modo TRK.




Na Figura \ref{fig:bank_modos} é apresentado o fluxo de operação dos modos do DCO. Cada modo possui um range de frequência de operação, e a mudança de uma unidade no $OTW$ define o menor passo de frequência,  $\Delta f$ possível em cada modo . Este passo pode ser calculado para cada modo conforme a equação \ref{eq:step_freq_mode}, considerando o range de frequência e o numero de bits em cada um. 
\begin{equation}
	\Delta f_{LSB_{mode}} = \frac{F.R_{mode}}{2^{b_{mode}}}
	\label{eq:step_freq_mode}
\end{equation}

\begin{figure}[h!]
	\caption{Fluxo de operação dos modos do DCO }
	\begin{center}
		\includegraphics[scale=0.6]{img/bank_modos.png}
	\end{center}
	\fonte{Adaptado de \citeonline{staszewski2006all}.}
	\label{fig:bank_modos}
\end{figure}

%
%Ao iniciar o DCO, o modo PVT é ativado e inicia no centro do range de ajuste com um valor de $OTW=127$, quando o valor é configurado em $0$ corresponde a menor frequência possível e $255$ a mais alta. Ele  executa a compensação de variações de processo, temperatura e tensão, centralizando a frequência do oscilador de forma mais grosseira com variações entre $1$ a $2MHz$.
%
%No segundo modo, ACQ, a frequência é ajustada para dentro da banda de interesse, com um passo de frequência médio $\Delta f_A$ com uma resolução de $\pm 500kHz$ variando $OTW$ entre 0 à 255.
%
%Por fim, o modo TRK possui a melhor resolução de ajuste, o que permite ajustar com precisão a frequência de saída do oscilador necessária para a modulação do sinal. Para permitir ajustes muito finos como $\pm 1kHz$, $OTW$ pode ser um valor fracionário neste modo. Para isto é utilizado uma topologia similar a do \textit{Fractional-N PLL}, permitindo uma melhor resolução com a implementação de um modulador Sigma Delta (SDM). SDMs são muito utilizados em conversores ADs, analógico para digital, por meio de um modulador com sobre amostragem seguido por um filtro digital/decimador que juntos produzem um sinal de alta resolução \cite{sdmtexas}.
Ao iniciar o DCO, todos os modos são iniciados no centro do range de ajuste, com um valor de $OTW=127$ para PVT e ACQ e $OTW=33$ para TRK. Configurar $OTW$ como $0$ corresponde à menor frequência possível, enquanto $255$ representa a frequência mais alta. O modo PVT é o primeiro a ser ajustado fazendo a compensação de variações de processo, temperatura e tensão, centralizando a frequência do oscilador de forma mais grosseira, apresentando variações entre $1$ a $2MHz$.

No segundo modo, ACQ, a frequência é ajustada para dentro da banda de interesse com um passo de frequência médio $\Delta f_A$, fornecendo uma resolução de $\pm 500kHz$ ao variar $OTW$ entre 0 e 255.

Por fim, o modo TRK possui a melhor resolução de ajuste, permitindo ajustar com precisão a frequência de saída do oscilador necessária para a modulação do sinal. Para permitir ajustes muito finos, como $\pm 1kHz$, $OTW$ pode ser um valor fracionário nesse modo. Para isso, é utilizada uma topologia semelhante à do \textit{Fractional-N PLL}, permitindo uma melhor resolução com a implementação de um modulador Sigma-Delta (SDM). Os SDMs são amplamente utilizados em conversores ADs (analógico para digital) por meio de um modulador com sobre-amostragem, seguido por um filtro digital/decimador, que juntos produzem um sinal de alta resolução \cite{sdmtexas}. Essa técnica de modulação é empregada para alcançar um ajuste preciso e refinado da frequência do DCO, garantindo a adequada modulação do sinal dentro do canal designado.


A Figura \ref{fig:lc_bank_capacitor} apresenta o tanque LC de um DCO projetado para aplicação \textit{Bluetooth}, com a implementação dos três bancos de capacitores. Nos bancos PVT e ACQ cada capacitor representado por um bit possui um mesmo valor conforme o banco que ele pertence, a soma deles em conjunto com o capacitor $C_0$ definirá a frequência. 
%$C_0$ é um capacitor de valor fixo que defini a máxima frequência do oscilador


\begin{figure}[h!]
	\caption{Tanque LC com banco de capacitores para aplicação \textit{Bluetooth} }
	\begin{center}
		\includegraphics[scale=0.6]{img/lc_bank_capacitor.png}
	\end{center}
	\fonte{Adaptado de \citeonline{staszewski2006all}.}
	\label{fig:lc_bank_capacitor}
\end{figure}


Nos bancos de capacitores PVT e ACQ o valor individual de cada capacitor é definido pela equação \ref{eq:c_delta}, na qual leva em consideração a frequência minima e máxima de operação de cada banco. O valor de $C_0$ é igual ao valor de $C_{min}$ do banco PVT, ou seja a máxima frequência do oscilador.
\begin{equation}
	C_{max} = \frac{1}{L} \big( \frac{1}{2 \pi f_{min}} \big)^2
	\label{eq:cmax}
\end{equation}

\begin{equation}
	C_{min} = \frac{1}{L} \big( \frac{1}{2 \pi f_{max}} \big)^2
	\label{eq:cmin}
\end{equation}

\begin{equation}
	\Delta C^{banco} = \frac{	C_{max} - C_{min}}{2}
	\label{eq:c_delta}
\end{equation}

Após o ajuste do banco PVT a capacitância total é obtida pela equação \ref{eq:c_A_total}, considerando o número de capacitores acionados considerando seus valores e a metade dos capacitores dos bancos ACQ e TRK acionados.

\begin{equation}
	C^P = C_0 + C^A_{half} + C^T_{half} + \sum_{k=0}^{7} \bar{d}_k \cdot 2^k * 	\Delta C^P
	\label{eq:c_P_total}
\end{equation}

Da mesma forma após o ajuste do banco ACQ a capacitância total é obtida pela equação \ref{eq:c_A_total} considerando o valor do banco PVT já ajustado.
\begin{equation}
	C^A = C_P - C^A_{half} + C^T_{half} + \sum_{k=0}^{7} \bar{d}_k \cdot 2^k * 	\Delta C^A
	\label{eq:c_A_total}
\end{equation}


No modo TRK os capacitores são rearranjados de forma diferente. Utilizando peso binário quando $OTW$ mudar de 31 para 32 fara com que 6 capacitores sejam alterados, causando bordas de ruído no sinal de saída, o que não é desejável visto que neste modo é importante manter a estabilidade para garantir a comunicação.   Desta forma no modo TRK cada capacitor possui valor unitário, ou seja a mudança de $OTW$ fara com que apenas um capacitor seja alterado, evitando ruídos indesejados. O valor de cada capacitor é calculado conforme a equação \ref{eq:c_unit_TRK}, considerando o range de frequência necessário e o número de bits utilizado no modo. 

\begin{equation}
	C_\mu = \frac{	C_{max} - C_{min}}{2^{bTRK}}
	\label{eq:c_unit_TRK}
\end{equation}

Por fim a capacitância total do tanque LC pode ser obtida pela equação \ref{eq:c_T_total} ao final do ajuste do banco TRK, sendo a soma das capacitâncias dos bancos anteriores com o atual. Desta forma $C^T = C_{tot}$ e pode ser utilizada a equação \ref{eq:f_LC} para estimar a frequência de saída do oscilador.
\begin{equation}
	C^T = C_A -  C^T_{half} + (2^{bTRK} - OTW) \cdot 	C_\mu
	\label{eq:c_T_total}
\end{equation}

%%%%%%%%%%%%%%%%%%%%%%%%%%%%%%%%%%%%%%%%%%%%%%%%%%%%%%%%%%%%%%%%%%%%%
\subsection{Loop Filter}
%%%%%%%%%%%%%%%%%%%%%%%%%%%%%%%%%%%%%%%%%%%%%%%%%%%%%%%%%%%%%%%%%%%%%%

%%%%%%%%%%%%%%%%%%%%%%%%%%%%%%%%%%%%%%%%%%%%%%%%%%%%%%%%%%%%%%%%%%%%%%
\subsection{TDC}
%%%%%%%%%%%%%%%%%%%%%%%%%%%%%%%%%%%%%%%%%%%%%%%%%%%%%%%%%%%%%%%%%%%%%%

%%%%%%%%%%%%%%%%%%%%%%%%%%%%%%%%%%%%%%%%%%%%%%%%%%%%%%%%%%%%%%%%%%%%%%
\section{Ruído de Fase}
%%%%%%%%%%%%%%%%%%%%%%%%%%%%%%%%%%%%%%%%%%%%%%%%%%%%%%%%%%%%%%%%%%%%%%


%%%%%%%%%%%%%%%%%%%%%%%%%%%%%%%%%%%%%%%%%%%%%%%%%%%%%%%%%%%%%%%%%%%%%%
\section{Trabalhos correlatos}
%%%%%%%%%%%%%%%%%%%%%%%%%%%%%%%%%%%%%%%%%%%%%%%%%%%%%%%%%%%%%%%%%%%%%%


%%%%%%%%%%%%%%%%%%%%%%%%%%%%%%%%%%%%%%%%%%%%%%%%%%%%%%%%%%%%%%%%%%%%%%

