% ----------------------------------------------------------
% ELEMENTOS PRÉ-TEXTUAIS
% ----------------------------------------------------------
% \pretextual

% ---
% Capa
% ---
\imprimircapa
% ---

% ---
% Folha de rosto
% (o * indica que haverá a ficha bibliográfica)
% ---
%\imprimirfolhaderosto*
\imprimirfolhaderosto
% ---



% ---
% Agradecimentos
% ---
\begin{agradecimentos}
Gostaria de expressar meus sinceros agradecimentos a todos que contribuíram para a realização deste trabalho. Em primeiro lugar, dedico este trabalho à minha amada família, com destaque especial para a minha falecida mãe, cujo apoio e encorajamento foram fundamentais ao longo desta jornada acadêmica.

À minha companheira Débora, quero expressar minha profunda gratidão. Sua presença constante ao meu lado, apoiando-me e compreendendo as inúmeras noites e finais de semana dedicados aos estudos, foi essencial para que eu pudesse alcançar esse objetivo.

Agradeço também aos professores e colegas que estiveram ao meu lado durante todos esses anos de curso. Seus conhecimentos compartilhados, orientações valiosas e troca de experiências contribuíram significativamente para o meu desenvolvimento acadêmico e me proporcionaram uma formação completa.

%Por fim, expresso minha gratidão a todos os que, direta ou indiretamente, contribuíram para a realização deste trabalho, seja através de apoio emocional, incentivo ou colaboração intelectual. Suas contribuições foram inestimáveis ​​e desempenharam um papel fundamental na conclusão deste TCC. Sou imensamente grato a cada um de vocês."


\end{agradecimentos}
% ---

% ---
% Epígrafe
% ---

\begin{comment}
Epígrafe (Não se escreve a palavra epígrafe). Elemento opcional. 
A epígrafe deve ser colocada após o agradecimento; trata-se de uma citação, seguida de indicação de autoria, relacionada à matéria tratada no corpo do trabalho. Deve ser “[...] elaborada conforme a NBR 10520 [...]. Podem também constar epígrafes nas folhas ou páginas de abertura das seções primárias” (ABNT, 2002, p. 7). 
A fonte da epígrafe deve sempre ser mencionada nas referências. 

Citação direta até 3 linhas deve estar entre aspas e em parágrafo normal (vá até a janela de Estilo - selecione - Parágrafo), se tiver mais de 3 linhas, deve ser recuada 4 cm da margem esquerda, com fonte menor que 12 e espaçamento entre linhas simples 
\end{comment}

\begin{epigrafe}
    \vspace*{\fill}
	\begin{flushright}
		\textit{“Nothing is better than reading, and gaining more and more knowledge.”  \\
%        \citeonline{adams2007hitchhiker}
			Stephen Hawking
}
	\end{flushright}
\end{epigrafe}
% ---

% ---
% RESUMOS
% ---

% resumo em português
\setlength{\absparsep}{18pt} % ajusta o espaçamento dos parágrafos do resumo
\begin{resumo}
  
 \textbf{Palavras-chave}: CMOS, ADPLL, python.
\end{resumo}

% resumo em inglês
\begin{resumo}[Abstract]
 \begin{otherlanguage*}{english}

   \vspace{\onelineskip}
 
   \noindent 
   \textbf{Keywords}: Energy Harvesting, low power, IoT, LP Wan.
 \end{otherlanguage*}
\end{resumo}


% ---
% inserir lista de ilustrações
% ---
\pdfbookmark[0]{\listfigurename}{lof}
\listoffigures*
\cleardoublepage
% ---

% ---
% inserir lista de quadros
% ---
\pdfbookmark[0]{\listofquadrosname}{loq}
\listofquadros*
\cleardoublepage
% ---

% ---
% inserir lista de tabelas
% ---
\pdfbookmark[0]{\listtablename}{lot}
\listoftables*
\cleardoublepage
% ---

% ---
% inserir lista de abreviaturas e siglas
% ---
\begin{siglas}
    \item[A/D] \textit{Conversor Analógico-Digital}
    \item[Anatel]   \textit{Agência Nacional de Telecomunicações}
    \item[DBPSK]    \textit{differential binary phase shift keying} (Modulação por chaveamento de deslocamento de fase binária diferencial)
    \item[CI]   \textit{Circuito integrado}
    \item[CMOS] \textit{Complementary metal–oxide–semiconductor}  (metal óxido semicondutor de simetria complementar)
    \item[CSV]  \textit{Comma-separated values} (Valores separados por vírgula)
    \item[DC]  \textit{Direct Current} (Corrente contínua)
    \item[FLV] \textit{Frutas, Verduras e Legumes}
    \item[FSK] \textit{Frequency Shifting Keying} (Modulação por chaveamento de frequência)
    \item[GPIO]     \textit{General Purpose Input/Output} (Entradas e Saídas de uso Geral)
    \item[EH]   \textit{Energy Harvesting} (Coleta de Energia) 
    \item[ESR]  \textit{Equivalent series resistor} (Resistor série equivalente)
    \item[I$^2$C]  \textit{Inter-Integrated Circuit} (Comunicação Entre circuitos Integrados)
    \item[IOT]  \textit{Internet of Things} (Internet das coisas)
    \item[ISM]  \textit{industrial, scientific and medical} (Indústria, Ciência e Medicina)
    \item[LoRa] \textit{Long Range} (Longo alcance)
    \item[LPWAN] \textit{Low-power wide-area network} (Redes de longa distância de baixa potência)
    \item[M2M] \textit{machine-to-machine} (Máquina à máquina) 
    \item[MLP] \textit{Multilayer Perceptron} (Perceptron multicamadas)
    \item[PA]  \textit{Power Amplifier} (Amplificador de Potência)
    \item[PCI] \textit{Placa de circuito impresso}
    \item[RF]  \textit{Rádio Frequência}
    \item[RTC]  \textit{Real-time clock} (Relógio de tempo real)
    \item[TACO] \textit{Tabela Brasileira de Composição de Alimentos}
    \item[UNISINOS]  \textit{Universidade do Vale do Rio dos Sinos}

\end{siglas}
% ---

% ---
% inserir lista de símbolos
% ---
\begin{simbolos}
  \item[$t$] Tempo
  \item[$\tau$] Tau - Constante de tempo
  \item[$C$] Capacitor
  \item[$I$] Corrente
  \item[$V_{cc}$] Tensão da Fonte de corrente contínua
  \item[$\Delta V$] Diferença de tensão
  \item[$A$] Ampere
  \item[$F$] Faraday
\end{simbolos}
% ---

% ---
% inserir o sumario
% ---
\pdfbookmark[0]{\contentsname}{toc}
\tableofcontents*
\cleardoublepage
% ---