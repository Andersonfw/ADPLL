%%%%%%%%%%%%%%%%%%%%%%%%%%%%%%%%%%%%%%%%%%%%%%%%%%%%%%%%%%%%%%%%%%%%%%
\chapter{INTRODUÇÃO}\label{cap:introducao}
%%%%%%%%%%%%%%%%%%%%%%%%%%%%%%%%%%%%%%%%%%%%%%%%%%%%%%%%%%%%%%%%%%%%%%
A evolução constante das tecnologias de comunicação sem fio impulsionou o desenvolvimento acelerado do setor de telecomunicações. A implementação do 5G, tem desempenhado um papel fundamental. O 5G é caracterizado por sua capacidade de conectar bilhões de dispositivos, proporcionando taxas de transferência extremamente altas, latência mínima e velocidades superiores a Gbps \cite{khanh2022wireless}.

Desde o surgimento da primeira geração de redes móveis (1G) na década de 1980, houve uma demanda crescente por maior velocidade e eficiência energética. O advento do 5G possibilitou a conectividade em larga escala por meio dos dispositivos IoT, também conhecidos como Internet das Coisas. A IoT representa um ecossistema interconectado de dispositivos, pessoas, plataformas, software e soluções que se comunicam através da Internet \cite{sinche2019survey}.

Essa combinação do 5G e dos dispositivos IoT resultou em um cenário em que tudo está conectado à Internet. De acordo com \cite{Cisco2020}, estima-se que até 2030 haverá cerca de 500 bilhões de dispositivos conectados em todo o mundo. A tecnologia IoT oferece uma infinidade de aplicações e soluções que abrangem diversos setores, como cidades inteligentes, agricultura inteligente, varejo inteligente, sistemas de transporte inteligentes e ecossistemas IoT \cite{khanh2022wireless}.

Eficiência energética é um dos principais requisitos de comunicações sem fio em aplicações IoT afim de estender a vida da bateria dos dispositivos. \cite{souza_2020_systemlevel}. A miniaturização dos circuitos CMOS, (Complementary Metal-Oxide-Semiconductor) \cite{khan2021nanoscale}, proporciona redução da potência e um crescimento do número de transistores por unidade de área, e assim, impulsionando o uso de estruturas digitais ao invés de totalmente analógicas \cite{ferreira2020review}. 

Nesse contexto, qualquer dispositivo de comunicação sem fio, como IoT, necessita de um circuito de radio frequência (RF) para fazer a transmissão e recepção de dados. O sintetizador de frequência é uma parte deste circuito, responsável pela geração do oscilador local tanto do transmissor como do receptor. O sintetizador de RF é umas das maiores dificuldades em comunicações sem fio, pois o sistema requer baixo custo, baixa consumo energético e baixa tensão enquanto deve atender aos requisitos de ruido de fase e modulação do protocolo \cite{staszewski2006all}.

PLLs (\textit{Phase-Locked Loops}) são amplamente utilizados como sintetizador de frequência, para gerar o oscilador local, sendo responsável pelo maior consumo de potência do dispositivo pois está ativo tanto na transmissão como na recepção. Desta forma, se faz necessário um sintetizador de RF de baixa potência.

É nesse contexto que o ADPLL (\textit{All-Digital Phase-Locked Loop}) desempenha um papel importante. O ADPLL é um circuito puramente digital que oferece baixo consumo de energia e ocupa uma área relativamente pequena no chip. Sua operação é baseada em algoritmos digitais que controlam diretamente os elementos do loop de fase, eliminando a necessidade de componentes analógicos. Essa abordagem inovadora torna o ADPLL uma solução atraente para sistemas de comunicação sem fio de baixo consumo.

Diante dessas considerações, o presente trabalho tem como objetivo desenvolver um ADPLL para utilização em dispositivos IoT com comunicação do tipo \textit{Bluetooth}.  Propõem-se um simulador baseado em eventos de borda utilizando a linguagem de programação Python onde, serão exploradas as características e o desempenho do ADPLL em diferentes cenários, a fim de avaliar que atenda as especificações do \textit{Bluetooth}.

%A monografia está organizada da seguinte forma: na seção 2, serão apresentados os conceitos fundamentais relacionados a PLLs, incluindo uma revisão sobre PLLs convencionais e PLLs fracionais. A seção 3 discutirá os princípios de funcionamento do ADPLL e as vantagens oferecidas por essa abordagem. Na seção 4, será detalhada a metodologia utilizada para o desenvolvimento do simulador de eventos. Os resultados e análises obtidos serão apresentados na seção 5, seguidos de conclusões e possíveis direções para trabalhos futuros na seção 6.



%%%%%%%%%%%%%%%%%%%%%%%%%%%%%%%%%%%%%%%%%%%%%%%%%%%%%%%%%%%%%%%%%%%%%%
\section{TEMA} 
%%%%%%%%%%%%%%%%%%%%%%%%%%%%%%%%%%%%%%%%%%%%%%%%%%%%%%%%%%%%%%%%%%%%%%

Desenvolver o escopo de um ADPLL para comunicação \textit{Bluetooth}, e efetuar uma simulação por meio de eventos de borda utilizando linguagem de programação Python para avaliação de atendimento dos requisitos.
%considerando parâmetros da tecnologia CMOS 65nm, e analisando a contribuição de ruido com diferentes parametrizações. 

%%%%%%%%%%%%%%%%%%%%%%%%%%%%%%%%%%%%%%%%%%%%%%%%%%%%%%%%%%%%%%%%%%%%%%
\section{DELIMITAÇÃO DO TEMA} 
%%%%%%%%%%%%%%%%%%%%%%%%%%%%%%%%%%%%%%%%%%%%%%%%%%%%%%%%%%%%%%%%%%%%%%
O trabalho encontra-se delimitado por:

\begin{itemize}
	\item Faixa de frequência \textit{Bluetooth} 2.4 a 2.45 GHz;
	\item Utilização de três bancos de capacitores no DCO;
	\item Simulação baseada em bordas de transições dos \textit{clocks};
	\item Parâmetros de transistores da tecnologia CMOS 65nm;
\end{itemize}

%%%%%%%%%%%%%%%%%%%%%%%%%%%%%%%%%%%%%%%%%%%%%%%%%%%%%%%%%%%%%%%%%%%%%%
\section{PROBLEMA}
%%%%%%%%%%%%%%%%%%%%%%%%%%%%%%%%%%%%%%%%%%%%%%%%%%%%%%%%%%%%%%%%%%%%%%
Dispositivos IoT requer uma maximização de sua vida útil por meio da diminuição de consumo energético. PLLs e N-PLLs requerem uma quantidade de energia significativa, desta forma é necessário uma forma mais eficiente energeticamente para gerar o oscilador local e que atenda a demanda de separação de canais de acordo com o protocolo.

%%%%%%%%%%%%%%%%%%%%%%%%%%%%%%%%%%%%%%%%%%%%%%%%%%%%%%%%%%%%%%%%%%%%%%
\section{OBJETIVOS}
%%%%%%%%%%%%%%%%%%%%%%%%%%%%%%%%%%%%%%%%%%%%%%%%%%%%%%%%%%%%%%%%%%%%%%
O objetivo é o desenvolvimento de um escopo e análise comportamental dos blocos que compõem um ADPLL e então, simular para verificar que atenda aos critérios de ruido e desvio de frequência para o protocolo \textit{Bluetooth}.
%%%%%%%%%%%%%%%%%%%%%%%%%%%%%%%%%%%%%%%%%%%%%%%%%%%%%%%%%%%%%%%%%%%%%%
\section{OBJETIVOS ESPECÍFICOS}
%%%%%%%%%%%%%%%%%%%%%%%%%%%%%%%%%%%%%%%%%%%%%%%%%%%%%%%%%%%%%%%%%%%%%%
\begin{itemize}
	\item estudar o comportamento dos blocos que compõem o ADPLL;
	\item estudar o impacto de cada bloco causa na saída do sistema em questão de ruido;
	\item simular um ADPLL completo por meio de simulação de eventos;
	\item avaliar o comportamento com diferentes parametrizações;
\end{itemize}
%%%%%%%%%%%%%%%%%%%%%%%%%%%%%%%%%%%%%%%%%%%%%%%%%%%%%%%%%%%%%%%%%%%%%%


%%%%%%%%%%%%%%%%%%%%%%%%%%%%%%%%%%%%%%%%%%%%%%%%%%%%%%%%%%%%%%%%%%%%%%








