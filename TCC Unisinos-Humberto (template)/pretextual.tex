% ----------------------------------------------------------
% ELEMENTOS PRÉ-TEXTUAIS
% ----------------------------------------------------------
% \pretextual

% ---
% Capa
% ---
\imprimircapa
% ---

% ---
% Folha de rosto
% (o * indica que haverá a ficha bibliográfica)
% ---
%\imprimirfolhaderosto*
\imprimirfolhaderosto
% ---



% ---
% Agradecimentos
% ---
\begin{agradecimentos}
Este trabalho é dedicado a minha família como forma de retribuição a todo esforço empenhado para garantir a minha educação ao longo de minha vida.

Com uma menção especial a minha esposa Camila e minhas duas filhas Isadora e Flávia que me ajudaram de várias maneiras até que a última palavra fosse inserida nesse trabalho.

Além é claro dos professores e coleguinhas que me acompanharam ao longo de todos os anos de curso e contribuíram para minha formação acadêmica de forma completa.

\end{agradecimentos}
% ---

% ---
% Epígrafe
% ---

\begin{comment}
Epígrafe (Não se escreve a palavra epígrafe). Elemento opcional. 
A epígrafe deve ser colocada após o agradecimento; trata-se de uma citação, seguida de indicação de autoria, relacionada à matéria tratada no corpo do trabalho. Deve ser “[...] elaborada conforme a NBR 10520 [...]. Podem também constar epígrafes nas folhas ou páginas de abertura das seções primárias” (ABNT, 2002, p. 7). 
A fonte da epígrafe deve sempre ser mencionada nas referências. 

Citação direta até 3 linhas deve estar entre aspas e em parágrafo normal (vá até a janela de Estilo - selecione - Parágrafo), se tiver mais de 3 linhas, deve ser recuada 4 cm da margem esquerda, com fonte menor que 12 e espaçamento entre linhas simples 
\end{comment}

\begin{epigrafe}
    \vspace*{\fill}
	\begin{flushright}
		\textit{“42 is the Answer to the Ultimate Question of Life, the Universe, and Everything.”  \\
        \citeonline{adams2007hitchhiker}
}
	\end{flushright}
\end{epigrafe}
% ---

% ---
% RESUMOS
% ---

% resumo em português
\setlength{\absparsep}{18pt} % ajusta o espaçamento dos parágrafos do resumo
\begin{resumo}
    A crescente demanda de dispositivos IoT vem acompanhada de uma carência tecnológica por baterias eficientes para garantir a flexibilidade dos sistemas que necessitam estar desconectados de cabos. Neste trabalho, foi explorado o desenvolvimento de um dispositivo IoT destinado a ser alimentado exclusivamente por um sistema de \textit{Energy Harvesting} que tem como princípio a utilização do meio para obter energia, tendo como sua fonte de acúmulo de energia um capacitor. Esta técnica tem como objetivo aproveitar a energia do meio em que o dispositivo se encontra como a: solar, cinética, térmica entre outras. Desta forma, os paradigmas que se apresentam giram em torno de uma comunicação de baixo consumo e um gerenciamento eficiente para armazenagem desta potência. Neste cenário, parte-se de alternativas como o HT32SX que foi construído para atender demandas de extremamente baixo consumo e rádio integrado que pode ser adaptado para diversas aplicações de transmissão de dados também com pouco consumo. Com isso, foi desenvolvido um dispositivo IoT com uma linha de comunicação capaz de ser independente de cabos ou baterias. Como resultado, foi possível executar 499 leituras, gravações na memória e transmissões ao longo de 20 dias sem a necessidade de recarga adicional. 


    %Este trabalho se propõe a desenvolver uma solução de módulo embarcado capaz de funcionar por longos períodos de tempo apenas com o uso de um super capacitor, coletando dados do ambiente como a temperatura, armazenando na memória e inclusive transmitindo esses dados através de uma comunicação de rádio frequência se for necessário.
    
 \textbf{Palavras-chave}: Energy Harvesting, low power, IoT, LP Wan.
\end{resumo}

% resumo em inglês
\begin{resumo}[Abstract]
 \begin{otherlanguage*}{english}
The growing demand for IoT devices is accompanied by a technological need for efficient batteries to ensure the flexibility of systems that need to be disconnected from cables.In this work, an IoT device will be developed to be powered exclusively by a Energy Harvesting system whose principle is to use the medium to obtain energy, having a capacitor as its source of energy accumulation.This technique aims to take advantage of the energy of the medium in which the device is located, such as: solar, kinetic, thermal, among others. In this way, the paradigms that are presented revolve around a low consumption communication and an efficient management for the storage of this power. In this scenario, we start with alternatives such as the HT32SX, which was built to meet the demands of extremely low consumption and an integrated radio that can be adapted for various data transmission applications also with little consumption. With this, it is expected to design an IoT device with a communication line capable of being independent of cables or batteries. As a result, it was possible to perform 499 reads, writes and transmissions over 20 days without the need for additional recharge.

   \vspace{\onelineskip}
 
   \noindent 
   \textbf{Keywords}: Energy Harvesting, low power, IoT, LP Wan.
 \end{otherlanguage*}
\end{resumo}


% ---
% inserir lista de ilustrações
% ---
\pdfbookmark[0]{\listfigurename}{lof}
\listoffigures*
\cleardoublepage
% ---

% ---
% inserir lista de quadros
% ---
\pdfbookmark[0]{\listofquadrosname}{loq}
\listofquadros*
\cleardoublepage
% ---

% ---
% inserir lista de tabelas
% ---
\pdfbookmark[0]{\listtablename}{lot}
\listoftables*
\cleardoublepage
% ---

% ---
% inserir lista de abreviaturas e siglas
% ---
\begin{siglas}
    \item[A/D] \textit{Conversor Analógico-Digital}
    \item[Anatel]   \textit{Agência Nacional de Telecomunicações}
    \item[DBPSK]    \textit{differential binary phase shift keying} (Modulação por chaveamento de deslocamento de fase binária diferencial)
    \item[CI]   \textit{Circuito integrado}
    \item[CMOS] \textit{Complementary metal–oxide–semiconductor}  (metal óxido semicondutor de simetria complementar)
    \item[CSV]  \textit{Comma-separated values} (Valores separados por vírgula)
    \item[DC]  \textit{Direct Current} (Corrente contínua)
    \item[FLV] \textit{Frutas, Verduras e Legumes}
    \item[FSK] \textit{Frequency Shifting Keying} (Modulação por chaveamento de frequência)
    \item[GPIO]     \textit{General Purpose Input/Output} (Entradas e Saídas de uso Geral)
    \item[EH]   \textit{Energy Harvesting} (Coleta de Energia) 
    \item[ESR]  \textit{Equivalent series resistor} (Resistor série equivalente)
    \item[I$^2$C]  \textit{Inter-Integrated Circuit} (Comunicação Entre circuitos Integrados)
    \item[IOT]  \textit{Internet of Things} (Internet das coisas)
    \item[ISM]  \textit{industrial, scientific and medical} (Indústria, Ciência e Medicina)
    \item[LoRa] \textit{Long Range} (Longo alcance)
    \item[LPWAN] \textit{Low-power wide-area network} (Redes de longa distância de baixa potência)
    \item[M2M] \textit{machine-to-machine} (Máquina à máquina) 
    \item[MLP] \textit{Multilayer Perceptron} (Perceptron multicamadas)
    \item[PA]  \textit{Power Amplifier} (Amplificador de Potência)
    \item[PCI] \textit{Placa de circuito impresso}
    \item[RF]  \textit{Rádio Frequência}
    \item[RTC]  \textit{Real-time clock} (Relógio de tempo real)
    \item[TACO] \textit{Tabela Brasileira de Composição de Alimentos}
    \item[UNISINOS]  \textit{Universidade do Vale do Rio dos Sinos}

\end{siglas}
% ---

% ---
% inserir lista de símbolos
% ---
\begin{simbolos}
  \item[$t$] Tempo
  \item[$\tau$] Tau - Constante de tempo
  \item[$C$] Capacitor
  \item[$I$] Corrente
  \item[$V_{cc}$] Tensão da Fonte de corrente contínua
  \item[$\Delta V$] Diferença de tensão
  \item[$A$] Ampere
  \item[$F$] Faraday
\end{simbolos}
% ---

% ---
% inserir o sumario
% ---
\pdfbookmark[0]{\contentsname}{toc}
\tableofcontents*
\cleardoublepage
% ---